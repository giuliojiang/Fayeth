%%%%%%%%%%%%%%%%%%%%%%%%%%%%%%%%%%%%%%%%%
% University/School Laboratory Report
% LaTeX Template
% Version 3.1 (25/3/14)
%
% This template has been downloaded from:
% http://www.LaTeXTemplates.com
%
% Original author:
% Linux and Unix Users Group at Virginia Tech Wiki 
% (https://vtluug.org/wiki/Example_LaTeX_chem_lab_report)
%
% License:
% CC BY-NC-SA 3.0 (http://creativecommons.org/licenses/by-nc-sa/3.0/)
%
%%%%%%%%%%%%%%%%%%%%%%%%%%%%%%%%%%%%%%%%%

%----------------------------------------------------------------------------------------
%	PACKAGES AND DOCUMENT CONFIGURATIONS
%----------------------------------------------------------------------------------------

\documentclass{article}

\usepackage[version=3]{mhchem} % Package for chemical equation typesetting
\usepackage{siunitx} % Provides the \SI{}{} and \si{} command for typesetting SI units
\usepackage{graphicx} % Required for the inclusion of images
\usepackage{natbib} % Required to change bibliography style to APA
\usepackage{amsmath} % Required for some math elements 

\usepackage{geometry}
 \geometry{
 a4paper,
 total={170mm,257mm},
 left=20mm,
 top=20mm,
 }

\setlength\parindent{0pt} % Removes all indentation from paragraphs

\renewcommand{\labelenumi}{\alph{enumi}.} % Make numbering in the enumerate environment by letter rather than number (e.g. section 6)

%\usepackage{times} % Uncomment to use the Times New Roman font

%----------------------------------------------------------------------------------------
%	DOCUMENT INFORMATION
%----------------------------------------------------------------------------------------

\title{Fayeth \\ DIMACS Fuzzer} % Title

\author{Giulio Jiang \\ Abraao Mota \\ Domenico Marino} % Author name

\date{\today} % Date for the report

\begin{document}

\maketitle % Insert the title, author and date

\begin{center}
\begin{tabular}{l r}
Imperial College London \\
Department of Computing \\
\end{tabular}
\end{center}

% If you wish to include an abstract, uncomment the lines below
% \begin{abstract}
% Abstract text
% \end{abstract}

\newpage



\section{Introduction}

Fayeth is a fuzzer targeted to boolean satisfiability solvers. It can generate and mutate 
random inputs and attempt to trigger bugs in the System Under Test (SUT), or create CNF files 
that maximize the total code coverage of the software.

\section{Overview}

Fayeth is implemented in Java, and it can be executed in two different modes: 
Undefined Behaviour Mode and Functional Error Mode. The model of operation is simple: 
once started, Fayeth will automatically generate test cases and evaluate their 
effectiveness on the target SAT solver. The fuzzer can be stopped at any time, 
and the output directories will contain the generated inputs that Fayeth considers 
the most interesting.

\section{Undefined Behaviour Mode}

The Undefined Behaviour Mode is aimed at finding undefined behaviour bugs, crashes and timeouts in the
SAT Solver being targeted. It generates automatically random inputs of different kinds,
and observes the behaviour of the SUT to try to detect whether some bug has manifested itself.
The Undefined Behaviour Mode's detection power is amplified by the use of the Address
Sanitizer to compile the SAT Solver, giving Fayeth the ability to detect a wider range of
potentially silent bugs.\\

Fayeth can generate a range of inputs, which we can categorize as follows:

\begin{itemize}

\item \verb|Random Strings| - A completely random sequence of characters, which an extremely low probability of representing a valid CNF formula. Random Strings will target early phases of CNF parsing.

\item \verb|Random Clauses| - TODO

\item \verb|Random Valid CNF| - TODO

\end{itemize}

\subsection{Strategies}

\subsection{}

\section{Functional Error Mode}

\subsection{Strategies}

\section{Software Architecture}

\section{Optimizations}

\section{Evaluation}




\end{document}